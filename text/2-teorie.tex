\chapter{Teoretický základ}
\label{2-teorie}

V této kapitole jsou objasněny způsoby monitorování radiační situace, popsán sběr dat a představen výstupní report.

\section{Ionizující záření}

Ionizující záření je tok buď hmotných částic, nebo fotonů elektromagnetického záření, které mají schopnost ionizovat atomy prostředí či excitovat jejich jádra. Příčinou vzniku může být radioaktivní rozpad, kosmické záření nebo jej lze vytvořit uměle. Je průvodním jevem jaderných procesů, během nichž se jádro či obal atomu dostávají do energeticky nestabilního excitovaného stavu. Pro návrat do stabilního stavu musí vyzářit energii ve formě částic či fotonů elektromagnetického záření. Podle způsobu interakce s prostředím se ionizující záření dělí na dva druhy:

\begin{itemize}
	\item \textbf{Přímo ionizující záření} mohou způsobovat jen nabité částice, které mají dostatečnou kinetickou energii k vyvolání ionizace. Mezi nabité částice patří např. elektrony, částice $\alpha$, $\beta$.
	
	\item \textbf{Nepřímo ionizující záření} je tvořeno nenabitými částicemi (neutrony, fotony), které prostředí přímo sami neionizují, ale při vzájemném působení s prostředím předávají svou kinetickou energii sekundárním, nabitým částicím, jež následně přímými účinky na atomy látku ionizují.
\end{itemize}

% http://astronuklfyzika.cz/JadRadFyzika6.htm
% http://fbmi.sirdik.org/1-kapitola/13/131.html

\subsection{Fyzikální veličiny a jednotky}

\begin{itemize}
	\item \textbf{Plošná aktivita}
	
	Množství radioaktivní látky je charakterizováno aktivitou. Jedná se o počet radioaktivních přeměn vztažených na jednotku času. Aktivita má za jednotku becquerel {[}Bq{]}. U plošných zdrojů záření se používá plošná aktivita neboli podíl aktivity a celkové plochy látky {[}Bq/m$^2${]}.
	
	\item \textbf{Dávkový příkon}
	
	Působení ionizujícího záření popisuje veličina dávka (či také absorbovaná dávka), která je určena jako poměr střední energie předané ionizujícím zářením látce o dané hmotnosti. Základní jednotkou je gray {[}Gy{]}, který odpovídá energii 1 joule absorbované v kilogramu látky. Dávkový příkon je přírůstek dávky v časovém intervalu {[}Gy/s{]}. V praxi se používají nižší jednotky {[}µGy/h{]} nebo {[}cGy/h{]}.

	\item \textbf{Příkon dávkového ekvivalentu}
	
	Na rozdíl od předchozích veličin patří dávkový ekvivalent mezi tzv. radiobiologické veličiny, tj. zohledňující účinky působení různých druhů záření na živou hmotu. Dávkový ekvivalent je součin dávky v uvažovaném bodě tkáně a jakostního činitele, jenž vyjadřuje rozdílnou biologickou účinnost různých druhů záření. Příkon fotonového dávkového ekvivalentu (\zk{PFDE}) je přírůstek dávkového ekvivalentu způsobeného fotony v časovém intervalu {[}Sv/s{]}. Prostorový dávkový ekvivalent odpovídá dávkovému ekvivalentu, který by uspořádané a rozšířené pole ionizujícího záření způsobilo v hloubce d = 10 mm v ICRU kouli\footnote{Koule z materiálu adekvátně odpovídajícího tkáni lidského těla o průměru 30 cm} na rádius vektoru opačného směru, než je orientované pole. Příkon prostorového dávkového ekvivalentu (\zk{PPDE}) je přírůstek prostorového dávkového ekvivalentu v časovém intervalu {[}Sv/s{]}.

% https://www.sujb.cz/aplikace/monras/?lng=cs_CZ#z3
%http://fbmi.sirdik.org/

		\begin{table}[h!]
			\centering
			\caption{Fyzikální veličiny}
			\label{tab:tab_veliciny}
			\begin{tabular}{|c|c|c|}
				\hline
				\textbf{Veličina}           			& \textbf{Jednotka}  & \textbf{Značka}  \\ \hline
				Aktivita								& becquerel			 & {[}Bq{]}			\\ \hline
				Plošná aktivita							& becquerel/m$^2$	 & {[}Bq/m$^2${]}	\\ \hline
				Dávka                      	 			& gray               & {[}Gy{]}         \\ \hline
				Dávkový příkon              			& gray/sekunda		 & {[}Gy/s{]}       \\ \hline
				Dávkový ekvivalent          			& sievert            & {[}Sv{]}         \\ \hline
				Příkon fotonového dávkového ekvivalentu & sievert/sekunda	 & {[}Sv/s{]}       \\ \hline
				Příkon prostorového dávkového ekvivalentu & sievert/sekunda	 & {[}Sv/s{]}       \\ \hline
			\end{tabular}
		\end{table}
\end{itemize}

\section{Monitorování radiační situace}

Pod monitorováním radiační situace je myšleno pravidelné sledování úrovně ionizujícího záření v okolním prostředí, měření obsahu umělých radionuklidů ve složkách životního prostředí a potravních řetězců, ale také sledování radioaktivity v těle člověka. \cite{monras}

Systematické plošné monitorování radiační situace má na území České republiky počátky v dubnu 1986, kdy došlo k havárii v \zk{JE} Černobyl. Je zajišťováno pomocí celostátní Radiační monitorovací sítě (\zk{RMS}) spravované Státním ústavem pro jadernou bezpečnost (\zk{SÚJB}).

\zk{RMS} běžně operuje v tzv. normálním režimu, v případě mimořádné radiační situace přechází do tzv. havarijního režimu. Během normálního režimu pracují stálé složky \zk{RMS}, jež v první řadě zajišťují provoz fixních měřících míst a analýzu dat z nich získaných. Mezi stálé složky \zk{RMS} patří v první řadě \zk{SÚJB}, Státní ústav radiační ochrany (\zk{SÚRO}) a Český hydrometeorologický úřad (\zk{ČHMÚ}). Při přechodu do havarijního režimu dochází rovněž k aktivaci pohotovostních složek. Po zvážení je zahájen radiační průzkum, který sestává z dalšího monitorování na měřících bodech, pojezdového měření a v případě potřeby i měření leteckého. \cite{suro} Všechny získané informace jsou potřebné pro rozhodování o opatřeních vedoucích ke snížení nebo odvrácení ozáření.

% (1) http://www.mocr.army.cz/informacni-servis/zpravodajstvi/chemici-se-specialisty-statniho-ustavu-radiacni-ochrany-spolecne-monitorovali-radiaci-z-vrtulniku-118918/

% (2) https://www.suro.cz/cz/rms	

\subsection{Sběr dat}	

Mezi hlavní způsoby monitorování radiační situace patří:

\begin{itemize}
	\item \textbf{napevno umístěné detektory (měřící body)}
	
\begin{figure}[H]
    \centering
      \includegraphics[width=150pt]{./pictures/03_merici-sonda-v-libereckych-kasarnach_2.jpg}
      \caption[Měřící sonda libereckých chemiků]{Měřící sonda libereckých chemiků
      (autor: \href{http://www.acr.army.cz/informacni-servis/zpravodajstvi/armadni-radiacni-monitorovaci-site-nacvicoval-zasah-pri-radiaci-131355/}{kapitán Ing. Jakub Šimíček})}
      \label{fig:sonda}
\end{figure}
	
	Jedním ze způsobů hodnocení radiační situace je zjištění odchylek od dlouhodobého průměru \zk{PFDE}, resp. \zk{PPDE}. Dlouhodobě měřené hodnoty \zk{PFDE} na území České republiky se pohybují mezi 0,1 až 0,2 {[}µSv/h{]}\footnote{Zdroj: https://www.sujb.cz/aplikace/monras/}. Tato měření jsou nepřetržitě prováděna na pevně umístěných detektorech.
	
	Základním systémem, umožňujícím průběžné sledování radiační situace na území ČR, je Síť včasného zjištění (\zk{SVZ}) spravovaná Regionálními centry (RC) \zk{SÚJB}, \zk{SÚRO}, \zk{ČHMÚ} a Armádou ČR. SVZ je v okolí a uvnitř areálu jaderných elektráren Dukovany a Temelín doplněna teledozimetrickými systémy (\zk{TDS}), jejichž činnost je zajišťována ČEZ, a.s. Detekční jednotky \zk{SVZ} i \zk{TDS} obsahují dva detektory s různým rozsahem měření veličiny \zk{PFDE}. Dalším způsobem zjištění odchylek od průměru jsou integrální měření fotonových, resp. prostorových dávkových ekvivalentů zjišťovaná v měřících místech s integrálními dozimetry, které tvoří teritoriální síť a lokální síť v okolí \zk{JE}.
	
		
	\item \textbf{pozemní monitorování}
	
\begin{figure}[H]
    \centering
      \includegraphics[width=250pt]{./pictures/02_mereni-radiace-v-terenu-z-vozidla_2.jpg}
      \caption[Liberecký chemik měří radiaci v terénu ve speciálním vozidle]{Liberecký chemik měří radiaci v terénu ve speciálním vozidle
      (autor: \href{http://www.acr.army.cz/informacni-servis/zpravodajstvi/armadni-radiacni-monitorovaci-site-nacvicoval-zasah-pri-radiaci-131355/}{kapitán Ing. Jakub Šimíček})}
      \label{fig:pozemni}
\end{figure}
	
	Sběr dat při pojezdovém měření je prováděn z vozidla jedoucího rychlostí 40 km/h po určené trase. Spolu s měřenou hodnotou se zaznamenává čas a poloha měření.
	
	Při naměření předem stanoveného dávkového příkonu osádka vozu již dále nepokračuje ve směru rostoucích hodnot do epicentra výbuchu. \uv{Takováto úroveň se pouze vytyčí a souřadnice jejího naměření se zahlásí radiostanicí na sběrné stanoviště (veliteli jednotky radiačního průzkumu, popřípadě na analytickou skupinu),} popisuje nadporučík Jiří Komárek, starší důstojník Skupiny monitorování a leteckého průzkumu 314. centra výstrahy \zk{ZHN}. Následně se osádka vrací zpět po stejné trase. 
	
	Právě závislost pozemního průzkumu na trasách přesunu, tj. cestách, a vystavení osádky vyšším hodnotám ionizujícího záření je jeho největší nevýhodou. Proto se provádí jako doplněk k měření leteckému. Hlavním zdrojem informací se stává ve chvíli, kdy povětrnostní podmínky nedovolí realizovat letecké monitorování.
	
	Pozemní monitorování zajišťují \zk{SÚJB}, \zk{SÚRO}, Hasičský záchranný sbor ČR, Generální ředitelství cel, Armáda ČR, Policie ČR a ČEZ, a.s. 
	
	\item \textbf{letecké monitorování}
	
	Letecké monitorování je prováděno z vrtulníku letícího ve výšce asi 100 m nad terénem po předem určených trasách. Naměřené údaje jsou přepočítány na úroveň radiace ve výšce 1 m nad terénem. Oproti pojezdovému měření si letecký průzkum může dovolit prozkoumat kontaminovaný prostor více do hloubky. Úkolem specialistů na palubě však je sledování měřených dávkových příkonů, aby eventuálně mohli buď upravit parametry průzkumu (rychlost, výška letu), nebo změnit trasu letu. Cílem je rychlé, orientační zmapování velké oblasti bez ohledu na charakter terénu. 
	
	\uv{V případě jaderného výbuchu by mohl být letecký průzkum potenciálně využit ke zmapování radioaktivní stopy, kterou takový výbuch po vypadání částic zanechá. Důležité jsou tady samozřejmě vhodně zvolené podmínky průzkumu,} doplňuje dále nadporučík Komárek. 
	
	Omezujícími podmínkami leteckého průzkumu je povětrnostní situace a doba, po kterou je třeba vyčkat, než vypadají radioaktivní částice na zemský povrch. Během čekání lze předběžně určit orientační dávkové příkony v epicentru vztažené na odhad mohutnosti výbuchu, na jejichž základě se rozhodne o provedení průzkumu ve vhodném časovém horizontu (po "vymření" krátkodobých radionuklidů, kdy radiace v epicentru poklesne). Následně lze provést průzkum v souladu s principem \zk{ALARA}, tedy že dávka ionizujícího záření, které je osoba vystavena, má být tak nízká, jaké lze rozumně dosáhnout.
	
	Letecké monitorování v ČR provádí \zk{SÚRO} a Armáda ČR. 

\end{itemize}
	
	Nově získané hodnoty radiačního monitorování jsou při vkládání do programu \zk{MonRaS} porovnávány s informačními úrovněmi. Informační úrovně existují dvě: 1. a 2. IU. Při jejich překročení jsou zjišťovány důvody jejich překročení a případně provedeny kroky nutné k odstranění příčiny.
	
\subsection{Armádní radiační monitorovací síť}	
	
% zdroj: JK
% http://www.vvubrno.cz/userstorage/files/pdf/prospekty/svz-arms-sit-vcasneho-zjisteni.pdf
% http://www.acr.army.cz/informacni-servis/zpravodajstvi/armadni-radiacni-monitorovaci-site-nacvicoval-zasah-pri-radiaci-131355/
	
	U Armády České republiky se monitorováním radiační situace zabývají dvě jednotky. 
	 
	 314. centrum výstrahy proti zbraním hromadného ničení (\zk{ZHN}) v Hostivici je zodpovědné především za letecký radiační průzkum a monitorování radiační situace pomocí Sítě včasného zjištění patřící do Armádní radiační monitorovací sítě (\zk{SVZ ARMS}). Jedná se o soustavu 16 stacionárních sond (původně jich bylo 17, ale sonda v Rakovníku byla zrušena a dosud nebyla nahrazena). \zk{AČR} svými daty ze \zk{SVZ ARMS} přispívá do celostátní \zk{RMS}, kterou spravuje \zk{SÚJB}.
	 
	 K provedení leteckého radiačního průzkumu AČR využívá vrtulník Mil Mi-17. V současné době AČR disponuje gama spektrometrickým systémem IRIS (Integrated Radiation Information System), přístrojem MobDOSE a palubním detektorem DP-3a. IRIS umí měřit nejen dávkový příkon a dávku, ale především energetické spektrum detekovaného záření gama, tj. lze jej využít ke gama spektrometrii.
	
	 Za pozemní průzkum jsou zodpovědné především jednotky 31. pluku radiační, chemické a biologické ochrany v Liberci. Sběr dat při pojezdovém měření je prováděn z vozidla Land Rover LR-110, lehkého obrněného kolového transportéru BRDM-2 nebo džípu UAZ-469 (postupně vyřazován). Přístroje používané k měření dávkového příkonu jsou DP-98, AS-67 nebo DP-3b a jsou pevně spojené s vozidlem. 
	 
	 
\section{Textový report}
% JK
% https://systematic.com/defence/products/a/military-messaging/app-11-and-adatp-3/	

APP-11 je katalog, který specifikuje formát textových zpráv (\zk{MTF}) používaných v \zk{NATO} ke komunikaci se spřátelenými silami. Poslední verze katalogu APP-11 obsahuje více než 400 zpráv pokrývajících každý aspekt operací NATO, při němž se vyskytuje potřeba předávání informací pomocí standardizovaných zpráv. Zprávy jsou sestaveny na základně pravidel uvedených v technické publikaci ADatP-3. 

\zk{MTF} zprávy se skládají ze dvou částí - hlavičky a těla. Hlavička zahrnuje data o původci, příjemci či klasifikaci. Tělo pak obsahuje předávanou informaci ve formátu specifikovaném v APP-11. 

\begin{figure}[H]
    \centering
      \includegraphics[width=300pt]{./pictures/Military-Messaging-white-borders-988px.png}
      \caption[Struktura zprávy MTF]{Struktura zprávy MTF
      (zdroj: \href{https://systematic.com/defence/products/a/military-messaging/app-11-and-adatp-3/}{SYSTEMATIC})}
      \label{fig:systematic}
  \end{figure}
  
Hlavní výhody \zk{MTF} zprávy:

\begin{itemize}
	\item \textbf{jasně strukturovaný obsah je pochopitelný a předchází nedorozuměním} 
	\item \textbf{je přenositelná mezi různými národy a systémy} 
	\item \textbf{při přenosu požaduje malou šířku vlnového pásma}
	\item \textbf{je vhodná pro strojové zpracování}
\end{itemize}

Výstupem ze zásuvného modulu je textový report ve formátu v souladu s katalogem APP-11. Jedná se o seznam souřadnic lomových bodů polygonů v systému \zk{MGRS}, který bude součástí \zk{MTF} zprávy. (viz Kapitola 4.4 Výstupní report)

\subsection{Military grid reference system}
Hlásný systém MGRS (\zk{MGRS}) je systém udávání polohy používaný Severoatlantickou aliancí (\zk{NATO}). Využívá Mercatorovo příčné válcové konformní zobrazení (\zk{UTM}), případně \zk{UPS}. 

Na rozdíl od jiných souřadnicových systémů, které vyjadřují polohu pomocí dvojice hodnot (šířka/délka, x/y), MGRS využívá jen jednu hodnotu a to alfanumerický řetězec znaků. Ten je tvořen třemi údaji:

\begin{itemize}
	\item \textbf{označení zóny}
	
	Jedna zóna je tvořena sférickým čtyřúhelníkem referenčního elipsoidu, jenž je vymezen zeměpisnými poledníky a rovnoběžkami. Sférické čtyřúhelníky vznikají rozdělením povrchu Země do 60 poledníkových zón o šířce 6°, které jsou následně děleny ve směru rovnoběžek na 19 vrstev po 8° a 1 vrstvu o výšce 12°.
	
	Poledníkové pásy jsou číslovány od 1 do 60 od obrazu poledníku 180° z. d. směrem na východ. Vrstvy jsou značeny velkými písmeny latinské abecedy C-X (s vynecháním písmen I a O vzhledem k jejich podobnosti s číslicemi) od obrazu rovnoběžky 80° j. š. na sever.
	
	Jedinečné označení zóny je složeno z čísla poledníkového pásu následovaného písmenem rovnoběžkového pásu (např. 33U).
	
	\item \textbf{označení čtverce 100 x 100 km}
	
	Jednotlivé zóny jsou rozděleny na čtverce o hraně 100 km sítí čar rovnoběžných s obrazem příslušného osového poledníku a rovníku. Jelikož se poledníkové pásy směrem k pólům zužují, zóny obsahují určitý počet úplných čtverců a na krajích neúplné čtverce o proměnlivé šířce. 
	
	Pro označení sloupců jsou použita písmena A-Z (s vynecháním I a O), značení začíná u obrazu poledníku 180° z. d. a pokračuje směrem na východ, po písmenu Z se celá řada opět opakuje. Vrstvám jsou přidělena písmena A-V (bez I a O). První vrstva lichých poledníkových pásů je značena písmenem A, u sudých pásů začíná písmenem F. Po písmenu V se abeceda opakuje. 
	
	Označení čtverce se skládá ze dvou písmen - označení sloupce a vrstvy (např. VR)
		
	\item \textbf{souřadnice bodu ve 100 km čtverci}
	
	V rámci čtverce je upřesněna poloha bodu za pomoci n+n číslic, kde první sada číslic určuje východní souřadnici od levého kraje čtverce a druhá sada severní souřadnici od okraje spodního. Podle přesnosti vyjádření polohy bodu n nabývá hodnot 1, 2, 3, 4 nebo 5. 
	
		\begin{itemize}
				\item 1+1 číslice pro souřadnici s přesností 10 km (\textit{54})
				\item 2+2 číslice pro souřadnici s přesností 1 km (\textit{5748})
				\item 3+3 číslice pro souřadnici s přesností 100 m (\textit{577484})
				\item 4+4 číslice pro souřadnici s přesností 10 m (\textit{57704840})
				\item 5+5 číslic pro souřadnici s přesností 1 m (\textit{5770048400})
		\end{itemize}	
		 
\end{itemize}

\begin{figure}[H]
    \centering
      \includegraphics[width=250pt]{./pictures/MGRS_tvorba.png}
      \caption[Postup tvorby souřadnic MGRS]{Postup tvorby souřadnic MGRS
      (zdroj: \href{https://www.maptools.com/tutorials/mgrs/quick_guide}{MapTools})}
      \label{fig:maptools}
\end{figure}
  
Poloha Fakulty stavební ČVUT v Praze by tedy pomocí hlásného systému MGRS s přesností na metry byla vyjádřena řetězcem ...

Standardem NATO je rozlišení 10 m \cite{wiki}, Armáda ČR pro výstup zásuvného modulu požaduje přesnost na 1 m.

Východní a severní souřadnice v systému MGRS se vždy vztahují k levému dolnímu rohu čtverce. Při přechodu na nižší přesnost se souřadnice nezaokrouhlují, ale přebytečné číslice se odříznou, aby bylo zajištěno, že bod zůstane ve správném čtverci s nižší přesností.

% Souřadnice (například 33UVR577484, tj. 33U VR 577 484) se skládají z několika informací:



%zdroje: 
% https://www.vugtk.cz/slovnik/5479_hlasny-system-mgrs - základní definice
% http://uhulag.mendelu.cz/files/pagesdata/cz/geodezie/geodezie1/souradnicove_systemy.pdf (str. 48)
% http://www.diverzanti.cz/cl_36a - nejvíc
% https://en.wikipedia.org/wiki/Military_Grid_Reference_System
